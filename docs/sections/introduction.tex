\newpage
\section{项目简介}
项目地址: \href{https://github.com/wuc9521/rep-flow}{https://github.com/wuc9521/rep-flow} 

\subsection{项目背景}
本项目设计了一种基于对话系统引导的自动化测试结果复现技术,通过对自动化测试结果的有效分析与相关测试知识库的指导,对话系统可根据测试人员当前复现情况进行有效的引导,使其更高效地完成自动化测试结果复现的工作。

\subsection{项目结构}
\dirtree{%
    .1 \lstinline{rep-flow/}.
        .2 \lstinline{README.md}.
        .2 \lstinline{Makefile}: 项目的主要配置文件..
        .2 \lstinline{data/}.
            .3 \lstinline{README.md}.
            .3 \lstinline{apk}: 用于复现bug的apk文件.
            .3 \lstinline{corpus/}: 用于对话系统的语料库..
            .3 \lstinline{guidance/}: 存放复现需要的图片, 以bug编号分类..
            .3 \lstinline{list/}: 存放复现bug的动作列表, 以bug编号命名..
            .3 \lstinline{state/}: 存放用户当前状态的截图, 以时间戳命名..
        .2 \lstinline{log/}: 存放项目运行过程中生成的后端日志和脚本日志..
        .2 \lstinline{script/}.
            .3 \lstinline{README.md}.
            .3 \lstinline{__init__.py}.
            .3 \lstinline{config.json}: 设备的配置文件..
            .3 \lstinline{main.py}: 用于获取用户当前状态的脚本..
        .2 \lstinline{model/}..
            .3 \lstinline{__init__.py}.
            .3 \lstinline{common.py}:用于预处理图片..
            .3 \lstinline{hist.py}:三直方图算法模型实现与测试..
            .3 \lstinline{ssim.py}:结构相似性算法模型实现与测试..
            .3 \lstinline{process.py}:分析图片相似度的主程序..
        .2 \lstinline{static/}: 前端的静态文件..
        .2 \lstinline{index.html}: 前端的主页..
        .2 \lstinline{app.py}: 后端的主要逻辑..
        .2 \lstinline{test/}.
            .3 \lstinline{README.md}.
            .3 \lstinline{monkey.py}:使用monkey工具对目标应用执行随机测试.
            .3 \lstinline{script/}:随机测试复现和过程截图获取.
            .3 \lstinline{linkedlist.py}:负责数据格式转换,测试结果以链表形式存储备用. 
        .2 \lstinline{requirements.txt}: 涉及到的Python依赖.
        .2 \lstinline{package.json}: 涉及到的Node依赖.
        .2 \lstinline{utils/}: 涉及到的工具文件.
}
