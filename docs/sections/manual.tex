\section{运行、交互说明}

\subsection{工具基本信息说明}
\noindent 安卓版本: Android 13, Pixel 4 XL

\noindent 从运行和交互的角度, 主要分为自动测试和复现引导两个模块, 其中两个模块统一版本, 但相互独立, 可以分别运行, 分别对应测试人员和复现人员的需求。

\subsection{测试模块}
\noindent 测试人员首先需要保持Appium server启动
\begin{lstlisting}[language=sh, showtabs=true]
    appium
\end{lstlisting}
然后运行虚拟机或者连接手机,再新建终端运行/test路径下的monkey.py
\begin{lstlisting}[language=sh, showtabs=true]
    python monkey.py
\end{lstlisting}
如果遇到运行问题,参考README中的指引,可能需要更改python client的版本。如果要更改随机测试的目标或参数需要更改monkey.py中的相应参数

\subsection{复现引导模块}

我们为项目配置了完整的 \lstinline{Makefile}. 为了运行项目, 只需要在项目根目录下运行 \lstinline{make} 就可以看到详细的指示.

\scalebox{0.9}{
    \lstinputlisting[language=sh]{gist/make.txt}
}

\noindent 一般来说, 用户需要先接上手机或者运行虚拟机, 然后在项目根目录下运行

\begin{lstlisting}[language=sh]
    make install
\end{lstlisting}
来安装需要的Python和Node.js依赖. 然后运行
\begin{lstlisting}[language=sh, showtabs=true]
    make run
\end{lstlisting}
来启动对话系统, 模型和检测脚本. 

\scalebox{0.9}{
\lstinputlisting[language=sh]{gist/make-run.txt}
}


\noindent 目前我们只支持 Android 平台, 并且我们默认使用的是 Pixel 4 XL 的模拟器. 如果你想要使用其他的设备, 请修改 \lstinline{script/config.json} 中的相应变量.


% \lstinline{//TODO:} 如果ml部分还有需要配置的, 写在这个模块下就可@yc
% \subsubsection{e.g.: 图片比对功能配置}
